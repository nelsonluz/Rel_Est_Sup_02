\subsection{Histórico}
\hspace{1.5cm}
Segundo o site da Prefeitura, a cidade de Manaus tem sua criação no século XVII para no primeiro momento demonstrar e fixar a presença portuguesa na região amazônica. Tem seu núcleo urbano, localizado na margem esquerda do Rio negro, teve início com a construção do Forte da Barra de São José no ano de 1669. Ao redor do Forte de São José do Rio Negro se desenvolveu o povoado do Lugar da Barra, que por conta da sua posição geográfica passou a ser sede da Comarca do São José do Rio Negro. Em 1755, por meio de Carta régia, a antiga missão de Mariuá foi escolhida como capital, passando a se chamar vila de Barcelos, anos mais tarde a sede foi transferida para o Lugar da Barra, que em 1832 tornou-se Vila da Barra, e em 24 de outubro de 1848, a Cidade da Barra de São José do Rio Negro. No entanto, com a elevação da Comarca à categoria de Província, em 1850, a Cidade da Barra, passou a se chamar em 04 de setembro de 1856, Cidade de Manaus, tornando-se independente do Estado do Grão-Pará. O nome lembra a tribo indígena dos Manáos, que habitavam a região onde hoje é Manaus antes de serem extintos por conta da civilização portuguesa, e seu significado é “mãe dos deuses”.\\

\hspace{1.5cm}
A cidade de Manaus viveu o surto da economia gomífera, de 1870 à 1913. Com a da perda do mercado mundial para a borracha asiática, a mesma retornou a um novo período de isolamento até o advento da Zona Franca de Manaus, em 1970. Segundo PNUD (2013)\cite{pnud2013}, entre os anos de 2000 e 2010, seu crescimento populacional foi numa taxa de 2,51\% anualmente. Neste período sua taxa de urbanização foi de 99,36\% para  99,49\%.